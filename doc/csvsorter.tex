% \LaTeX-Main\
%% The csvsorter program - version 0.95 beta (2018/01/11)
%% csvsorter.tex: Manual
%%
%% Copyright (c) 2014-2018, Prof. Dr. Dr. Thomas F. Sturm
%% All rights reserved.
%%
%% Redistribution and  use in source  and binary forms, with or without
%% modification, are  permitted provided  that the  following conditions
%% are met:
%%
%% 1. Redistributions  of source  code must  retain the  above copyright
%% notice, this list of conditions and the following disclaimer.
%%
%% 2. Redistributions in binary form  must reproduce the above copyright
%% notice, this list  of conditions and the following  disclaimer in the
%% documentation and/or other materials provided with the distribution.
%%
%% 3. Neither  the name  of the  project's author nor  the names  of its
%% contributors may be used to  endorse or promote products derived from
%% this software without specific prior written permission.
%%
%% THIS SOFTWARE IS  PROVIDED BY THE COPYRIGHT  HOLDERS AND CONTRIBUTORS
%% "AS IS"  AND ANY  EXPRESS OR IMPLIED  WARRANTIES, INCLUDING,  BUT NOT
%% LIMITED  TO, THE  IMPLIED WARRANTIES  OF MERCHANTABILITY  AND FITNESS
%% FOR  A PARTICULAR  PURPOSE  ARE  DISCLAIMED. IN  NO  EVENT SHALL  THE
%% COPYRIGHT HOLDER OR CONTRIBUTORS BE  LIABLE FOR ANY DIRECT, INDIRECT,
%% INCIDENTAL, SPECIAL, EXEMPLARY,  OR CONSEQUENTIAL DAMAGES (INCLUDING,
%% BUT  NOT LIMITED  TO, PROCUREMENT  OF SUBSTITUTE  GOODS OR  SERVICES;
%% LOSS  OF USE,  DATA, OR  PROFITS; OR  BUSINESS INTERRUPTION)  HOWEVER
%% CAUSED AND  ON ANY THEORY  OF LIABILITY, WHETHER IN  CONTRACT, STRICT
%% LIABILITY, OR TORT (INCLUDING NEGLIGENCE OR OTHERWISE) ARISING IN ANY
%% WAY  OUT  OF  THE USE  OF  THIS  SOFTWARE,  EVEN  IF ADVISED  OF  THE
%% POSSIBILITY OF SUCH DAMAGE.
\documentclass[a4paper,11pt]{ltxdoc}

\usepackage[T1]{fontenc}
\usepackage[latin1]{inputenc}
\usepackage[english]{babel}
\usepackage{lmodern,parskip,array}
\usepackage{amsmath,amssymb}
\usepackage[svgnames,table,hyperref]{xcolor}
\usepackage{tikz}
\usepackage{varioref}
\usepackage[pdftex,bookmarks,raiselinks,pageanchor,hyperindex,colorlinks]{hyperref}
\usepackage{varwidth,cleveref,incgraph}
\usepackage{array,booktabs,csvsimple}

\usepackage[a4paper,left=2.5cm,right=2.5cm,top=1.5cm,bottom=1.5cm,
    marginparsep=5mm,marginparwidth=10mm,
    headheight=0mm,headsep=0cm,
    footskip=1.5cm,includeheadfoot]{geometry}
\usepackage{fancyhdr}
\fancyhf{}
\fancyfoot[C]{\thepage}%
\renewcommand{\headrulewidth}{0pt}
\renewcommand{\footrulewidth}{0pt}
\pagestyle{fancy}
\tolerance=2000%
\setlength{\emergencystretch}{20pt}%

\usepackage{array,tabularx}
\usepackage{amsmath}
\usepackage{lipsum}

\usepackage[cache]{minted}% minted 2.0

\usepackage{changepage}
\strictpagecheck

\usepackage[many,listings,minted]{tcolorbox}
\usetikzlibrary{arrows.meta}

\def\version{0.95 beta}%
\def\datum{2018/01/11}%
\def\docdatum{2018/01/11}

\hypersetup{
  pdftitle={Manual for the csvsorter program},
  pdfauthor={Thomas F. Sturm},
}


\tcbset{mysource/.style={
  enhanced,colback=yellow!10!white,colframe=red!50!black,
  coltitle=red!30!black,colbacktitle=red!50!yellow!20!white,
  fonttitle=\bfseries\sffamily,arc=1pt,
  drop lifted shadow=black!50!yellow,boxrule=0.6pt,
  attach boxed title to top center={yshift*=-\tcboxedtitleheight/2-0.3pt},
  varwidth boxed title*=-2cm,before title=\strut,
  boxed title style={enhanced,boxrule=0.6pt,
    frame code={ \path[tcb fill frame] ([xshift=-4mm]frame.west) -- (frame.north west)
    -- (frame.north east) -- ([xshift=4mm]frame.east)
    -- (frame.south east) -- (frame.south west) -- cycle; },
    interior code={ \path[tcb fill interior] ([xshift=-2.5mm]interior.west)
    -- (interior.north west) -- (interior.north east)
    -- ([xshift=2.5mm]interior.east) -- (interior.south east) -- (interior.south west)
    -- cycle;}  },
  listing engine=minted,
  minted language=xml,
  minted options={fontsize=\small,mathescape},
  minted style=colorful,
  listing only},
mycsv/.style={mysource,minted language=bat,
  colback=green!5!white,colframe=green!50!black,
  coltitle=green!30!black,colbacktitle=green!50!yellow!20!white},
  }


\NewTCBListing{xml}{ O{} }{  mysource, #1}

\NewTCBInputListing{\inputXML}{ O{} m }{mysource,listing file={#2},#1}

\NewTCBInputListing{\inputCSV}{ O{} m }{mycsv,listing file={#2},#1}

\NewTCBListing{csvtext}{ O{} }{mycsv,#1}

\tcbset{cmd/.style={verbatim,colupper=white,fontupper=\ttfamily\bfseries,
  colback=black!75!white,colframe=black}}

\lstdefinestyle{cmdline}{nolol,columns=fullflexible,
  aboveskip={0\p@ \@plus 6\p@},
  belowskip={0\p@ \@plus 6\p@},
  keepspaces=true,
  breaklines=true,
  breakatwhitespace=true,
  language=command.com,keywordstyle=\bfseries\color{red!35!white},
  basicstyle=\bfseries\ttfamily\color{white},
  moredelim={[is][\color{blue!35!white}\sffamily\slshape]{�}{�}}
  }

\DeclareTotalTCBox{\command}{ s O{} v }
  {cmd,before upper=\strut,#2}
  {\IfBooleanTF{#1}{\textcolor{red}{\ttfamily\bfseries > }}{}%
   \lstinline[style=cmdline]^#3^}

\DeclareTotalTCBox{\comopt}{ O{} v }{cmd,colupper=blue!35!white,fontupper=\bfseries\sffamily\slshape,#1}{#2}

\DeclareTotalTCBox{\comname}{ O{} v }{cmd,#1}{#2}

\DeclareTColorBox{optionbox}{ O{} m m }{%
  enhanced,colbacktitle=black!75!white,colframe=black,fonttitle=\bfseries\ttfamily,coltitle=white,
  title={-#2 {\color{blue!35!white}\sffamily\slshape #3}},
  attach boxed title to top left={yshift=-2mm},boxed title style={cmd}}

\DeclareTotalTCBox{\telement}{ O{} m }{enhanced,on line,size=small,boxrule=0.2pt,
  fontupper=\ttfamily,colupper=green!40!black,colback=green!10!white,colframe=green!70!black,
  #1}{<#2>}

\def\element#1{\texorpdfstring{\telement{#1}}{<#1>}}


\DeclareTotalTCBox{\attribute}{ O{} m m }{enhanced,on line,size=small,boxrule=0.2pt,
  fontupper=\ttfamily,colupper=blue!40!black,colback=blue!7!white,colframe=blue!70!black!50!white,
  #1}{#2=\textcolor{blue!25!red}{\detokenize{"}#3\detokenize{"}}}

\newcommand{\sign}[1]{\textcolor{blue!25!red}{\ttfamily\detokenize{"}#1\detokenize{"}}}

\makeatletter
\DeclareTColorBox[auto counter]{tabularbox}{ O{} m m }{enhanced,capture=hbox,size=tight,
  arc=1mm,auto outer arc,boxrule=0.5mm,
  title={\thetcbcounter: #2},label={#3},center title,colframe=blue!25!red!50!black,
  colbacktitle=blue!25!red!75!black,
  fonttitle=\bfseries,toptitle=1mm,bottomtitle=1mm,
  colback=red!5!white,drop lifted shadow,center,
  before upper=\arrayrulecolor{blue!25!red!50!black},
  float=tb,#1}
\makeatother

\DeclareTCBox{\versionbox}{ +O{} }{size=small,fontupper=\large\bfseries,
  colback=red!10!white,colframe=red!30!white,before=\par\bigskip,after=\par,#1}

\NewDocumentCommand{\csvsorter}{}{\textsf{\bfseries\color{red!20!black}CSV-Sorter}}

\newenvironment{changelist}{\begin{list}{}{%
  \renewcommand{\makelabel}[1]{\tcbox[on line,size=small,boxsep=1pt,
    colback=blue!50!black,colframe=blue!20!white,fontupper=\footnotesize\sffamily\bfseries,colupper=white]{\makebox[1.2cm]{\strut ##1}}\hfill}%
  \setlength{\labelwidth}{1.6cm}\setlength{\labelsep}{3mm}%
  \setlength{\leftmargin}{\labelwidth+\labelsep}%
  \setlength{\itemsep}{0mm}\setlength{\parsep}{2pt plus 2pt minus 0pt}%%
}}{\end{list}}

\setcounter{secnumdepth}{4}



%%%%%%%%%%%%%%%%%%%%%%%%%%%%%%%%%%%%%%%%%%%%%%%%%
\begin{document}

\begin{tcolorbox}[enhanced,center,hbox,tikznode,left=8mm,right=8mm,boxrule=0.4pt,
  colback=white,colframe=black!50!yellow,
  drop lifted shadow=black!50!yellow,
  before=\par\vspace*{5mm},after=\par\bigskip]
{\bfseries\LARGE The \csvsorter\ program}\\[3mm]
\scalebox{1.5}{\begin{tikzpicture}
\fill[top color=red!50!white,bottom color=blue!50!white,middle color=yellow!50!white] (-1,-0.7) rectangle (1,0.7);
\node[align=center,inner sep=0pt] (csv) {{\bfseries\Large C\kern-.01em\lower.5ex\hbox{S}\kern-.125em V}\\[-0.4ex]\bfseries Sorter};
\draw[red!10!white,rounded corners,line width=0.7mm,-{Stealth[length=3.1mm,line width=0.3mm,round]}] ([yshift=1mm,xshift=0.1mm]csv.west) -| ([xshift=-3mm,yshift=-6mm]csv.west)
  -- ([xshift=3mm,yshift=-6mm]csv.east) |- ([yshift=1mm,xshift=-0.1mm]csv.east);
\draw[blue!10!white,rounded corners,line width=0.7mm,-{Stealth[length=3.1mm,line width=0.3mm,round]}]
  ([yshift=3mm,xshift=-0.1mm]csv.east) -| ([xshift=3mm,yshift=6mm]csv.east) -- ([xshift=-3mm,yshift=6mm]csv.west)
  |-([yshift=3mm,xshift=0.1mm]csv.west);
\draw[red,rounded corners,line width=0.5mm,-{Stealth[length=2.5mm,line width=0.1mm,round]}] ([yshift=1mm]csv.west) -| ([xshift=-3mm,yshift=-6mm]csv.west)
  -- ([xshift=3mm,yshift=-6mm]csv.east) |- ([yshift=1mm]csv.east);
\draw[blue,rounded corners,line width=0.5mm,-{Stealth[length=2.5mm,line width=0.1mm,round]}]
  ([yshift=3mm]csv.east) -| ([xshift=3mm,yshift=6mm]csv.east) -- ([xshift=-3mm,yshift=6mm]csv.west)
  |-([yshift=3mm]csv.west);
\end{tikzpicture}}\\[3mm]
{\large Manual for version \version\ (\datum)}\\[3mm]
{\footnotesize Documentation \docdatum}\\[3mm]
\url{http://T-F-S.github.io/csvsorter/}
\end{tcolorbox}
\begin{center}
{\large Thomas F.~Sturm%
  \footnote{Prof.~Dr.~Dr.~Thomas F.~Sturm, Institut f\"{u}r Mathematik und Informatik,
    Universit\"{a}t der Bundeswehr M\"{u}nchen, D-85577 Neubiberg, Germany;
     email: \href{mailto:thomas.sturm@unibw.de}{thomas.sturm@unibw.de}} }
\end{center}

\begin{tcolorbox}[breakable,enhanced jigsaw,title={Contents},fonttitle=\bfseries\Large,
  colback=yellow!10!white,colframe=red!50!yellow!50!white,before=\par\bigskip\noindent,
  coltitle=red!30!black,colbacktitle=red!50!yellow!20!white,boxrule=0.4pt,arc=0pt,outer arc=0pt,
  attach boxed title to top center={yshift=-0.25mm-\tcboxedtitleheight/2,yshifttext=2mm-\tcboxedtitleheight/2},
  boxed title style={enhanced,boxrule=0.4pt},
  drop lifted shadow]
\columnseprule=0.4pt%
\columnsep=1cm%
\def\columnseprulecolor{\color{red!50!yellow!50!white}}%
\begin{multicols}{2}
\makeatletter
\@starttoc{toc}
\makeatother
\end{multicols}%
\end{tcolorbox}

\clearpage
\section{Introduction}
The \csvsorter\ program serves to sort
CSV\footnote{CSV file: file with comma separated values.} files.

It is a Java command-line program which
processes one CSV input file to one CSV output file
controlled by an XML configuration file. Depending on the configuration,
\csvsorter\  can deal with different formats, separators, delimiters,
various sorting presets, header and no header files.

The \csvsorter\  program was developed as external sorting program
for the \texttt{csvsimple} \LaTeX\ package,
see \url{http://www.ctan.org/tex-archive/macros/latex/contrib/csvsimple}.
But it can be used for any CSV sorting task.

\clearpage
\section{License}

\csvsorter\ is licensed under the
New BSD License\footnote{\url{http://opensource.org/licenses/BSD-3-Clause}}.
The New BSD License has been verified as a GPL-compatible free software license by the
Free Software Foundation\footnote{\url{http://www.fsf.org/}},
and has been vetted as an open source license by the
Open Source Initiative\footnote{\url{http://www.opensource.org/}}.

\bigskip

\begin{tcolorbox}[parbox=false,colframe=red!50!black,colback=yellow!10!white]
Copyright \copyright~2014-2018, Prof.~Dr.~Dr.~Thomas~F.~Sturm\\
All rights reserved.

Redistribution and use in source and binary forms, with or without modification,
are permitted provided that the following conditions are met:

\begin{enumerate}
\item Redistributions of source code must retain the above copyright notice, this
list of conditions and the following disclaimer.
\item Redistributions in binary form must reproduce the above copyright notice,
this list of conditions and the following disclaimer in the documentation and/or
other materials provided with the distribution.
\item Neither the name of the copyright holder nor the names of its contributors
may be used to endorse or promote products derived from this software without
specific prior written permission.
\end{enumerate}

\noindent\textsc{This software is provided by the copyright holders and
contributors ``as is'' and any express or implied warranties, including, but not
limited to, the implied warranties of merchantability and fitness for a
particular purpose are disclaimed. In no event shall the copyright holder or
contributors be liable for any direct, indirect, incidental, special, exemplary,
or consequential damages (including, but not limited to, procurement of
substitute goods or services; loss of use, data, or profits; or business
interruption) however caused and on any theory of liability, whether in
contract, strict liability, or tort (including negligence or otherwise) arising
in any way out of the use of this software, even if advised of the possibility
of such damage.}
\end{tcolorbox}




\clearpage
\section{Installation}

\csvsorter\  is written in Java and needs to have a
Java virtual machine (version 1.6 or higher) installed.
With some luck, you already have Java installed on your system.
To check this, open a command console i.\,e.\ some terminal on Linux
or executing \comname{cmd.exe} on Windows.

If executing \command*[before upper=]{java -version} on the command-line
displays some Java version information, you have
Java installed. Otherwise, you need to install a Java Runtime Environment to proceed.

\csvsorter\ is a portable program which needs not necessarily be installed in
some way. See Subsection~\ref{basiccommon} and Subsection~\ref{basicwindows}
how to execute \csvsorter. Further installation steps are found in Subsection~\ref{furtherinstallcommon}
and in Subsection~\ref{furtherinstallwindows}.

\subsection{Basic Execution (platform-independent)}\label{basiccommon}

The \csvsorter\  is an executable Jar-archive named \comname{csvsorter.jar}.
You can build \comname{csvsorter.jar} from the sources, but this is usually
not necessary.

Change with your command console to the directory, where you saved \comname{csvsorter.jar}.\\
Execute the following on the command-line in this directory:\\
\command*{java -jar csvsorter.jar}

You should see something like\\
\comname{This is CSV-Sorter} followed by\\
\comname{Configuration file is missing.} and further text.
This means that you are ready to use \csvsorter\ 
in the current directory.
If you get some Java errors, your installed virtual machine is probably too
old and you have to update it to use \csvsorter\ .


\subsection{Basic Execution (Windows)}\label{basicwindows}

The \csvsorter\  is wrapped into a Windows native executable named \comname{csvsorter.exe}.
It does not provide a GUI and therefore cannot be started by double-clicking.
Instead, open a command console with \comname{cmd.exe}, change with it
to the directory where you saved \comname{csvsorter.exe}.
Execute the program on the command-line in this directory:\\
\command*{csvsorter.exe}

You should see something like\\
\comname{This is CSV-Sorter} followed by\\
\comname{Configuration file is missing.} and further text.
This means that you are ready to use \csvsorter\ 
in the current directory.
If you get some Java errors, your installed virtual machine is probably too
old and you have to update it to use \csvsorter\ .

\clearpage
\subsection{Further Installation (somewhat platform-independent)}\label{furtherinstallcommon}
The further steps are optional and provide an easier use for \csvsorter\ .

Instead of calling \command*{java -jar csvsorter.jar �options�}, you are
recommended to create a shortcut batch/script to access \csvsorter\ 
easily from everywhere on your system.

On Windows, this would be something like:

\begin{xml}[title={csvsorter.cmd},minted language=bat]
@echo off
java -jar "c:\SOME PATH STRING\csvsorter.jar" %*
\end{xml}

Replace \comname{c:\SOME PATH STRING\csvsorter.jar} by the appropriate path
for \comname{csvsorter.jar}.

Put this \comname{csvsorter.cmd} to a directory which is part of your
system path. If you do not know where to put this file, create a directory
\comname{c:\bat}, put the file into it, and \emph{add} the new directory to
your path variable.

A similar script file is recommended for Linux users:

\begin{xml}[title={csvsorter},minted language=bash]
#!/bin/bash
java -jar "/SOME PATH STRING/csvsorter.jar" $@
# further:   chmod u+x csvsorter
\end{xml}
\catcode`\Q=14 Q$
\catcode`\Q=11

After these steps, \csvsorter\  is accessible system-wide by\\
\command*{csvsorter �options�}


\subsection{Further Installation (Windows)}\label{furtherinstallwindows}
Copy the \comname{csvsorter.exe} into a directory which is part of your
system path. If you intend to use it with the |csvsimple| \LaTeX\ package,
you could put it e.\,g.\ into the directory of the
\LaTeX\ binaries.


\clearpage
\section{Usage}
Depending on the installation done in the last section, \csvsorter\ 
is called by

\command*{java -jar csvsorter.jar �options�} or\\
\command*{csvsorter �options�} (used in the following)\par
where \comopt{options} are the following:

\begin{optionbox}{c}{configuration xml file}
  This mandatory configuration file controls the processing of the input CSV file.
  See the following examples and Section~\ref{sec:config} for details.
\end{optionbox}

\begin{optionbox}{l}{logfile}
  This file is used to write logging messages.
  If it is not specified or the configuration is faulty, \comname{csvsorter.log}
  is used instead.
\end{optionbox}

\begin{optionbox}{i}{input csv file}
  The CSV file to process.
\end{optionbox}

\begin{optionbox}{o}{output csv file}
  The CSV file to write after processing.
  It is an error to set the output file equal to the input file.
\end{optionbox}

\begin{optionbox}{x}{input=output csv file}
  The CSV file to process and to write after processing (overwriting).
\end{optionbox}

\begin{optionbox}{t}{token file}
  If the processing was successful, this file is written with content |\relax|.
  If there was no success, nothing happens.
  This file is used as token for interaction with the |csvsimple| \LaTeX package.
\end{optionbox}

\begin{optionbox}{q}{number}
  If the number is zero, terminal messages are printed. Otherwise,
  \csvsorter\ keeps quiet on the terminal, if no errors occur.
\end{optionbox}

\begin{itemize}
\item The presence of a configuration file is mandatory.
\item Note that the configuration file may contain the rest of the options.
\item Command-line options override configuration file settings.
\end{itemize}

\clearpage
\subsection{Example 1}
\inputCSV[title={songcontest.csv (input)}]{songcontest.csv}

This CSV file is to be sorted by the first column. Since the input
file has a header line, the name \comname{Title} for the first
column can be used inside the configuration file.

\inputXML[title={titlesort.xml (configuration)}]{titlesort.xml}

Processing:

\immediate\write18{csvsorter -c titlesort.xml -i songcontest.csv -o songcontest_sorted.csv -l csvsorter.csvlog}%
\command*{csvsorter -c titlesort.xml -i songcontest.csv -o songcontest_sorted.csv}

Output file:

\inputCSV[title={\detokenize{songcontest_sorted.csv (output)}}]{songcontest_sorted.csv}

Note that the configuration file can be used for any further examples independent
from the number of columns or the position of the \comname{Title} column.


\clearpage
\subsection{Example 2}
\inputCSV[title={songcontest.csv}]{songcontest.csv}

This CSV file is to be sorted by the \comname{Televote} numbers in
descending order. If candidates have the same Televote values, the
\comname{Country} is used for sorting:

\inputXML[title={televotesort.xml}]{televotesort.xml}

Processing:

\immediate\write18{csvsorter -c televotesort.xml -i songcontest.csv -o songcontest_sorted.csv -l csvsorter.csvlog}%
\command*{csvsorter -c televotesort.xml -i songcontest.csv -o songcontest_sorted.csv}

Output file:

\inputCSV[title={\detokenize{songcontest_sorted.csv}}]{songcontest_sorted.csv}

Note that the configuration file can be used for any further examples independent
from the number of columns or the position of the \comname{Televote}
and \comname{Country} columns.


\clearpage
\subsection{Example 3}
\inputCSV[title={songcontest.csv}]{songcontest.csv}

This CSV file is to be sorted by the sum of the
\comname{Televote} and \comname{Juryvote} numbers in
descending order. If candidates have the same values, the
\comname{Country} is used for sorting:

\inputXML[title={sumsort.xml}]{sumsort.xml}

Processing:

\immediate\write18{csvsorter -c sumsort.xml -i songcontest.csv -o songcontest_sorted.csv -l csvsorter.csvlog}%
\command*{csvsorter -c sumsort.xml -i songcontest.csv -o songcontest_sorted.csv}

Output file:

\inputCSV[title={\detokenize{songcontest_sorted.csv}}]{songcontest_sorted.csv}

Note that the configuration file can be used for any further examples independent
from the number of columns or the position of the \comname{Televote}, \comname{Juryvote},
and \comname{Country} columns.

\clearpage

The example is continued with a demonstration of on-the-fly sort in
\LaTeX\ documents using
the |csvsimple|\footnote{\url{www.ctan.org/tex-archive/macros/latex/contrib/csvsimple}}
package.

Instead of sorting the |songcontest.csv| example beforehand, it can be
sorted on-the-fly while compiling a \LaTeX\ document:

\csvset{csvsorter log=csvsorter.csvlog}
\begin{tcblisting}{mycsv,title={\LaTeX\ example},listing and text,
  minted language=latex,bicolor,colbacklower=white,fontlower=\small}
% \documentclass{article}
% \usepackage{array,booktabs,csvsimple}
% \begin{document}

\csvreader[sort by=sumsort.xml,
  head to column names,
  tabular=lllcc,
  table head=\toprule\textbf{Artist} & \textbf{Title} & \textbf{Country} &
             \textbf{Points} & \textbf{TV+JV}\\\midrule,
  table foot=\bottomrule]
{songcontest.csv}{}
{\Artist & \itshape''\Title'' & \Country &
 \bfseries\the\numexpr\Televote+\Juryvote\relax & $(\Televote+\Juryvote)$}

% \end{document}
\end{tcblisting}


\clearpage
\subsection{Example 4}
\inputCSV[title={songcontest.csv}]{songcontest.csv}

This CSV file is to be sorted by the sum of the
\comname{Televote} and \comname{Juryvote} numbers in
descending order. If candidates have the same values, the
\comname{Title} is used for sorting.
The output should contain only \comname{Juryvote}, \comname{Juryvote}, and
\comname{Title}. Further, we need double quotes as brackets and semicolons
as delimiters:

\inputXML[title={filtersort.xml}]{filtersort.xml}

Processing:

\immediate\write18{csvsorter -c filtersort.xml -i songcontest.csv -o songcontest_sorted.csv -l csvsorter.csvlog}%

\command*{csvsorter -c filtersort.xml -i songcontest.csv -o songcontest_sorted.csv}

Output file:

\inputCSV[title={\detokenize{songcontest_sorted.csv}}]{songcontest_sorted.csv}



\clearpage
\subsection{Example 5}
\inputCSV[title={gradetab.csv}]{gradetab.csv}

This CSV file is not to be sorted, but just to be reformatted.
The input uses tabulator signs as delimiters (invisible here).
The output should use commas and add additional curly braces:

\inputXML[title={format.xml}]{format.xml}

Processing:

\immediate\write18{csvsorter -c format.xml -i gradetab.csv -o gradetab_sorted.csv -l csvsorter.csvlog}%

\command*{csvsorter -c format.xml -i gradetab.csv -o gradetab_sorted.csv}

Output file:

\inputCSV[title={\detokenize{gradetab_sorted.csv}}]{gradetab_sorted.csv}


\clearpage

\section{Configuration}\label{sec:config}
The \csvsorter\  program is controlled by an XML configuration file
according to the following template.

\begin{xml}[title={Configuration file template}]
<?xml version="1.0" encoding="UTF-8"?>
<csv>
  <noHeader/>
  <delimiter sign=";"/>
  <bracket leftsymbol="doublequote" rightsymbol="doublequote" empty="false" />
  <outDelimiter signsymbol="comma"/>
  <outBracket left="{" right="}" empty="false"/>
  <transform/>
  <contentToLaTeX/>
  <charset in="InputCharset" out="OutputCharset"/>
  <language iso="de"/>
  <sortlines>
    <column name="Country" order="ascending" type="string"/>
    <column name="Points" order="ascending" type="integer"/>
    <column number="2" order="descending" type="string"/>
    <sum order="ascending" type="integer">
      <column name="Value"/>
      <column number="7"/>
    </sum>
  </sortlines>
  <outColumns>
    <column name="Points"/>
    <column name="Country"/>
  </outColumns>
  <input file="InputFile"/>
  <output file="OutputFile"/>
  <log file="LogFile"/>
</csv>
\end{xml}

Nearly all tag elements are optional and there is no specific order of appearance.
The document element
\element{csv} is mandatory.

\subsection{\element{noHeader}}
If this element is present, the CSV file(s) do not have a header line.
Note that in this case the columns can be addressed by number only.
If this element is not present, the first line of the CSV file is
interpreted as header line and its contents can be used to address
columns by names.

\subsection{\element{delimiter}}
This element defines the delimiter sign for the input file. If it is not
present, the comma is the default delimiter. The actual delimiter sign
is defined by one of two feasible attributes of the element.

\begin{description}
\item[\attribute{sign}{,}] The value of the attribute is the actual delimiter sign.
\item[\attribute{signsymbol}{comma}] The value of the attribute is
  a symbolic description of the actual delimiter sign.
  See Table~\ref{symbols} on page \pageref{symbols} for a list of feasible symbol names.
\end{description}

\begin{tabularbox}{Symbolic sign names}{symbols}
\begin{tabular}{>{\ttfamily}l|l}
\bfseries\rmfamily Description & \bfseries Sign\\\hline
braceleft & \sign{\{} curly brace left\\
braceright& \sign{\}} curly brace right\\
comma & \sign{,} comma\\
doublequote& \sign{\detokenize{"}} double quote\\
pipe& \sign{\detokenize{|}} pipe\\
semicolon& \sign{;} semicolon\\
singlequote& \sign{'} single quote\\
tab& tabulator sign\\
\end{tabular}
\end{tabularbox}


\subsection{\element{bracket}}
This element defines the bracket signs for entries of the input file. If it is not
present, double quotes are uses as default brackets.
The actual bracket signs are defined by the following feasible attributes of the element.

\begin{description}
\item[\attribute{left}{\{}] The value of the attribute is the actual bracket sign.
\item[\attribute{leftsymbol}{doublequote}] The value of the attribute is
  a symbolic description of the actual bracket sign.
  See Table~\ref{symbols} on page \pageref{symbols} for a list of feasible symbol names.
\item[\attribute{right}{\}}] The value of the attribute is the actual bracket sign.
\item[\attribute{rightsymbol}{doublequote}] The value of the attribute is
  a symbolic description of the actual bracket sign.
  See Table~\ref{symbols} on page \pageref{symbols} for a list of feasible symbol names.
\item[\attribute{empty}{true}] If the value of the attribute is |true|,
  no input brackets are used at all. Setting |left| and |right| to an empty
  string is \emph{not} equivalent to this (actually, this would be ignored)!
\end{description}

If the brackets are not set empty, brackets still are not mandatory
to be used in the input file. But if an opening bracket is found in the input file,
there has to be a matching closing bracket.


\subsection{\element{outDelimiter}}
This element defines the delimiter sign for the output file. If it is not
present, the input delimiter sign is used for the output also.
The actual delimiter sign is defined by one of two feasible attributes of the element.

\begin{description}
\item[\attribute{sign}{,}] The value of the attribute is the actual delimiter sign.
\item[\attribute{signsymbol}{comma}] The value of the attribute is
  a symbolic description of the actual delimiter sign.
  See Table~\ref{symbols} on page \pageref{symbols} for a list of feasible symbol names.
\end{description}

\clearpage
\subsection{\element{outBracket}}
This element defines the bracket signs for entries of the output file. If it is not
present, the input bracket signs are used as default brackets.
The actual bracket signs are defined by the following feasible attributes of the element.

\begin{description}
\item[\attribute{left}{\{}] The value of the attribute is the actual bracket sign.
\item[\attribute{leftsymbol}{doublequote}] The value of the attribute is
  a symbolic description of the actual bracket sign.
  See Table~\ref{symbols} on page \pageref{symbols} for a list of feasible symbol names.
\item[\attribute{right}{\}}] The value of the attribute is the actual bracket sign.
\item[\attribute{rightsymbol}{doublequote}] The value of the attribute is
  a symbolic description of the actual bracket sign.
  See Table~\ref{symbols} on page \pageref{symbols} for a list of feasible symbol names.
\item[\attribute{empty}{true}] If the value of the attribute is |true|,
  no output brackets are used at all. Setting |left| and |right| to an empty
  string is \emph{not} equivalent to this (actually, this would be ignored)!
\end{description}

If the lines of the output file are not needed to be transformed and this
element is not present, the output file is written with the same lines as
the input file (even with missing brackets).

\subsection{\element{transform}}
If this element is present, the input lines are always transformed to
output lines. Actually, output brackets are always set.

\subsection{\element{contentToLaTeX}}
If this element is present, the content text is processed to be more
\LaTeX\ friendly. Especially,
\sign{\textbackslash} is replaced by \sign{\textbackslash textbackslash\{\}},
\sign{\&} is replaced by \sign{\textbackslash\&\{\}}, etc.
You should not process a file twice with this setting!

\subsection{\element{charset}}
This element defines the character set for the input and output files.
If this element is not present, the default character set of the current
Java virtual machine is used depending upon the locale and character set of the
underlying operating system.

\begin{description}
\item[\attribute{in}{windows-1252}]
  This defines the character set for the input file.
\item[\attribute{out}{UTF-8}]
  This defines the character set for the output file. If an input character set
  is given but no output character set, then the
  input character set is used as output character also.
\end{description}

Feasible charset names are listed in the IANA Charset
Registry\footnote{\url{http://www.iana.org/assignments/character-sets/character-sets.xhtml}},
but not all of them will be implemented in the current Java virtual machine.
Of interest may be \sign{US-ASCII}, \sign{UTF-8} (Unicode),
\sign{windows-1252} (Windows Western Latin, e.\,g.\ German),
\sign{IBM850} (DOS-Latin-1).

\clearpage
\subsection{\element{language}}
This element defines the language used for sorting, e.\,g.\ for proper
observance of German umlauts, etc.
If this element is not present, the default locale of the current
Java virtual machine is used depending upon the locale of the
underlying operating system.

\begin{description}
\item[\attribute{iso}{de}]
The value of the attribute is the actual ISO 639 alpha-2 or alpha-3 language code,
e.\,g.\ \sign{de} (German), \sign{en} (English), \sign{fr} (French).
\end{description}

\clearpage
\subsection{\element{sortlines}}
This element defines the actual sorting presets. It may contain a list
of \element{column} and/or \element{sum} elements for a hierarchical
sorting specification. The first sub-element has the highest priority.

\subsubsection{\element{column}}
Defines a sorting rule according to one column. The column has to be
denoted by \texttt{name} or \texttt{number} and to be set to a \texttt{type}
of data.
\begin{description}
\item[\attribute{name}{NAME}]
  Denotes the column by a name given by the header line.
  The value is not case sensitive.
\item[\attribute{number}{1}]
  Denotes the column by number (started at 1).
\item[\attribute{order}{ascending}]
  Sets the sorting rule to be \sign{ascending} (default) or \sign{descending}.
\item[\attribute{type}{integer}]
  Sets the data type.
  See Table~\ref{datatypes} on page~\pageref{datatypes} for a list of feasible data types.
  If this attribute is not present, \sign{string} is used as data type.
\item[\attribute{default}{VALUE}]
  Sets a default value for the column content. It is used for sorting, if the actual
  column content cannot be parsed according to the given data type.
  If this attribute is not present, lines with unparsable content are ignored.
\end{description}

\begin{tabularbox}{Data types for columns}{datatypes}
\begin{tabular}{>{\ttfamily}l|l}
\bfseries\rmfamily Type & \bfseries Comment\\\hline
integer & integer value between $-2^{31}$ and $2^{31}-1$\\
long    & long value between $-2^{63}$ and $2^{63}-1$\\
double  & double precision floating point value (not localized!)\\
date    & date localized by \element{language}\\
string  & text localized by \element{language}
\end{tabular}
\end{tabularbox}


\subsubsection{\element{sum}}
Defines a sorting rule according to the summarized value of columns.
The columns a denoted by a list of embedded \element{column} elements.
Further, the sum has the following attributes:
\begin{description}
\item[\attribute{order}{ascending}]
  Sets the sorting rule to be \sign{ascending} (default) or \sign{descending}.
\item[\attribute{type}{integer}]
  Sets the mandatory data type.
  Feasible data types are \sign{integer}, \sign{long}, and \sign{double} only.
\item[\attribute{default}{VALUE}]
  Sets a default value for the sum. It is used for sorting, if the actual
  sum cannot be computed for any reasons.
  If this attribute is not present, lines with uncomputable sums are ignored.
\end{description}

\paragraph{\element{column}}
Defines one column of the sum. The column has to be
denoted by \texttt{name} or \texttt{number}.
\begin{description}
\item[\attribute{name}{NAME}]
  Denotes the column by a name given by the header line.
\item[\attribute{number}{1}]
  Denotes the column by number (started at 1).
\end{description}

\clearpage
\subsection{\element{outColumns}}
Defines a set of columns to be used for the output file.
The columns a denoted by a list of embedded \element{column} elements.

\subsubsection{\element{column}}
Defines one column of the output. The column has to be
denoted by \texttt{name} or \texttt{number}.
\begin{description}
\item[\attribute{name}{NAME}]
  Denotes the column by a name given by the header line.
\item[\attribute{number}{1}]
  Denotes the column by number (started at 1).
\end{description}

\subsection{\element{input}}
This element defines the input file.
The mandatory attribute of the element is:
\begin{description}
\item[\attribute{file}{FILENAME}] The value of the attribute is the actual file name.
\item[\attribute{overwrite}{true}] If this attribute is present and its value equals |true|,
  the input file is allowed to be overwritten. Note that you have to specify
  the output file nonetheless. Command-line options may change this setting.
\end{description}
The appropriate command-line option overwrites this value.

\subsection{\element{output}}
This element defines the output file.
The mandatory attribute of the element is:
\begin{description}
\item[\attribute{file}{FILENAME}] The value of the attribute is the actual file name.
\end{description}
The appropriate command-line option overwrites this value.

\subsection{\element{log}}
This element defines the log file.
The mandatory attribute of the element is:
\begin{description}
\item[\attribute{file}{FILENAME}] The value of the attribute is the actual file name.
\end{description}
The appropriate command-line option overwrites this value.


\clearpage
\section{Hints, Tricks, and Troubleshooting}

\subsection{Hierarchical brackets}
The line scanning algorithm tries to identify the columns of the CSV file
on a best-effort base. Therefore, it always uses the configured
\element{delimiter} \emph{and} \element{bracket}.
\begin{itemize}
\item Brackets can be omitted in the input file, but if an opening bracket was used,
  there has to be a matching closing bracket.
\item Everything between a delimiter and an opening bracket is considered
  a whitespace!
  Analogously, everything between a closing bracket and a delimiter is considered
  a whitespace!\par
  With standard setting,
\begin{csvtext}
..., bla " My text" bla,...
\end{csvtext}
  is interpreted as
\begin{csvtext}
...," My text",...
\end{csvtext}

\item Inside a bracket pair, a delimiter sign is interpreted as normal text.
\begin{csvtext}
...,"one,two",...
\end{csvtext}
Here, the single column content is \texttt{one,two}.

\item Brackets are interpreted hierarchically, i.\,e.\ you can have brackets
  inside brackets. Note that \emph{all} opening brackets need to have
  matching closing brackets. If the left bracket sign is identical to the
  right bracket sign (as in the standard case), detection is done on a
  best-effort base.
\begin{csvtext}
...,"one "two", and "three"",...
\end{csvtext}
Here, the single column content is \texttt{one "two", and "three"}.

\item With the standard settings, the following line is faulty, even if
  brackets are omitted:
\begin{csvtext}
...,Fl\"ache,...
\end{csvtext}
To circumvent the problem, you should configure other bracket signs or empty brackets,
even if you do not use brackets directly in the input file.

\end{itemize}




\clearpage
\section{Version History}

%\versionbox[before=\par]{Version \version\ (\datum)}
\versionbox[before=\par]{Version 0.95 beta (2018/01/11)}
\begin{changelist}
\item[changed] Windows launcher |csvsorter.exe| can be used for Java 9 now.
\item[changed] Java 7 or newer is required now.
\item[fixed]   Deprecated constructors replaced.
\item[fixed]   Documentation corrected for \element{noHeader}.
\end{changelist}

\versionbox{Version 0.94 beta (2014/07/14)}
\begin{changelist}
\item[new]     Token file for interaction with |csvsimple| with new command-line option 't'.
\item[new]     New command-line option 'q' to set quiet state.
\end{changelist}

\versionbox{Version 0.93 beta (2014/07/11)}
\begin{changelist}
\item[new]     Windows native executable wrapping the Jar-archive added.
\item[new]     More log messages added.
\end{changelist}

\versionbox{Version 0.92 beta (2014/07/09)}
\begin{changelist}
\item[fixed]   Data loss, if input and output file are the same, corrected.
\item[fixed]   Descending string sorting corrected.
\item[changed] Input and output files are checked to be different.
\item[changed] 'type' attribute is not mandatory any more (set to 'string' if not present).
\item[new]     Input=output file with new 'x' command-line option or 'overwrite' attribute.
\item[new]     Console messages added for error cases.
\item[new]     'default' attributes for columns and sums added.
\item[new]     Data type 'long' added.
\end{changelist}

\versionbox{Version 0.91 beta (2014/07/05)}
\begin{changelist}
\item[changed] Hierarchical bracket algorithm improved.
\item[changed] Speed optimization for brackets (about 50 percent).
\item[new]     Empty input and output brackets implemented.
\end{changelist}

\versionbox{Version 0.90 beta (2014/06/30)}
\begin{changelist}
\item[new] First public release.
\end{changelist}

\versionbox{Version unpublished (2008/12)}
\begin{changelist}
\item[new] Unpublished private version(s).
\end{changelist}

\end{document}
